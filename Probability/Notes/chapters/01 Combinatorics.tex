\chapter{Combinatorics}
\section{Permutations}
We have n objects, in how many ways we can pick/order them (order matters):
\begin{equation*}
    n!=n*(n-1)*(n-2)...*1
\end{equation*}
\medskip
We have n objects, but some objects are equal:
\begin{equation*}
    \frac{n!}{r_1!r_2!...}
\end{equation*}
where $r_1, r_2...$ are the number of repetitions of same objects (look at anagram example).
\section{Dispositions}
From a group n, pick k objects (order matters, we can choose each object once):
\begin{equation*}
    \frac{n!}{(n-k)!}=n*(n-1)*...*(k+1)
\end{equation*}
Example: 10 objects, 3 slots: $\#dispositions = 10*9*8 = \frac{10!}{7!}$\\
\medskip
From a group n, pick k objects (order matters, we can choose each object multiple times):
\begin{equation*}
    n^k = n*n*...*n \text{ (repeated k times)}
\end{equation*}
\section{Combinations}
From a group n, pick k objects (order doesn't matter, we can pick each object once):
\begin{equation*}
    \binom{n}{k} = \frac{n!}{k!(n-k)!}
\end{equation*}\\
\medskip
From a group n, pick k objects (order doesn't matter, we can choose each object multiple times):
\begin{equation*}
    \frac{(n+k-1)!}{k!(n-1)!}
\end{equation*}
\section{Examples}
Multiply possibilities of first case with those of second case etc.
\begin{example}
Un dipartimento di statistica decide di assegnare ai propri 25 laureati tre premi di diversa tipologia. Se ciascuno dei laureati potesse ricevere al massimo un premio, quante assegnazioni sarebbero possibili?\\
$\#E = 25*24*23$
\end{example}
\medskip
First permutate outer group, then inner group.
\begin{example}
Il Signor Amadori deve sistemare 10 libri in un ripiano della scaffalatura. Quattro libri sono di matematica, tre sono di chimica, due sono di storia e uno è di grammatica. Amadori, che è un tipo ordinato, vuole fare in modo che i libri sullo stesso argomento siano vicini in libreria. In quanti modi ciò si può realizzare?\\
$\#E = 4!*4!*3!*2!$
\end{example}
\medskip
Anagrams: $\#E = \frac{(\text{num. of letters})!}{(\text{num. of repeted letter A})!*(\text{num. of repeted letter B})!}$
\begin{example}
Quanti sono gli anagrammi di PEPPER?\\
$\#E = \frac{6!}{3!*2!}$
\end{example}
\medskip
Pick k elements in n. Order doesn't matter.
\begin{example}
Dieci ragazzi devono formare 2 squadre A e B di 5 membri ciascuna. Quante sono lo suddivisioni possibili?\\
$\#E = \binom{10}{5}=\frac{10!}{5!5!}$
\end{example}
If you have probelms asking "at least", ofter it is easier to compute the complement and then subtract.
\begin{example}
    Il sito dedicato al calcolo delle probabilità "cdp.com" richiede ai suoi utenti di registrarsi con una password. Le regole per la costruzione della password sono le seguenti:
    \begin{itemize}
        \item deve essere lunga esattamente 5 caratteri;
        \item lettere maiuscole e minuscole sono considerate distinte (la password è case sensitive);
        \item deve contenere almeno una lettera (non importa se maiuscola o minuscola) e almeno un simbolo (punto . oppure underscore);
        \item le lettere possibili sono quelle dell'alfabeto inglese (26 lettere);
        \item sono consentiti solamente lettere maiuscole o minuscole, il punto (.) e l'underscore (\_).
    \end{itemize}
    \#charachers = $26*2+2=52$\\
    From the total number of passwords($54^5$), I subtract the ones having only letters ($54^5$) and the ones having only symbols ($2^5$), so I get $n=54^4-52^5-2^5$.\\
    I can write cdp in $2^3$=8 ways. I can place cdp in 3 ways: cdp**, *cdp*, **cdp. In those 2 ** spots i can place 2 symbols ($2^2$ possibilities) or 1 symbol and 1 character ($52*2*2$ possibilities). The final result is:
    $3*2^3*(2^2+52*2^2)$ (permutations of string cdp, permutations of letters cdp, possible ways of writing **).
\end{example}