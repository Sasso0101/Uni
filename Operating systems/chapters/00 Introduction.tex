\chapter{Introduction}

An operating system is a program that acts as an intermediary between the user and the hardware. The main goals of an operating system are to be:
\begin{itemize}
    \item User side: friendly, reliable, safe, fast
    \item System side: easy to design, modular, error-free, flexible, efficient
\end{itemize}

An operating system is composed of a kernel, system programs and user programs. The \bold{kernel} is the core of the operating system, has complete control over everything that is happening in the system and runs for the whole time the system is turned on. System programs are other programs that are shipped with the operating system.

\section{Computer startup}

The first program that runs on startup is the \italics{bootstrap program}. This program is stored in the ROM or EEPROM and is usually called
\bold{firmware}. It initializes registers, memories and device controllers and loads the kernel into the main memory. The kernel starts \bold{daemons}, i.e. background processes that provide various services to the user. Examples of daemons in Linux systems are \code{systemd} (daemon that starts other daemons), \code{syslogd} (logging daemon) and \code{sshd} (serves SSH connections).

\section{Interrupts}

A general computer architecture is the following: there are one or more CPUs and device controllers that are connected through a common bus with a shared memory.

\image{images/Computer architecture.png}{10cm}{General computer architecture}

The CPU and IO devices work independently, but they need to communicate with
each other. They can achieve this using interrupts. For example, when a device
controller has finished some operation (such as loading data into a register),
it can inform the CPU that the data is ready to be read by generating an
interrupt.

\image{images/Interrupts.png}{8cm}{Interrupt timing diagram}

When an interrupt happens, the OS saves the current program counter (PC) and jumps to the routine that is responsible of handling the interrupt. The list of routines and its addresses are stored in the interrupt table. The table is initialized at startup and is stored in RAM for fast access.

A trap or exception is a software-generated interrupt caused by an error or an user request.

Depending on the importance of the interrupt, some interrupts must be handled immediately, while others can wait. The first ones are called non-maskable, while the latter are maskable.

While transferring data, the CPU receives an interrupt when every chunk of data has been successfully received. When a lot of data is transferred at once, a lot of interrupts are generated. To reduce the overhead, a feature called DMA (Direct Memory Access) has been introduced. Using this technique, the CPU receives an interrupt only after all the data has been transferred successfully.

\section{Storage}

Storage systems are categorized by speed, cost and volatility. Each storage systems has therefore its advantages and disadvantages, therefore there is no "best" storage device. Therefore a computer has multiple types of storage.

The primary memory is DRAM (dynamic random access memory, based on charged capacitors) or SRAM (static random access memory, based on inverters), which is usually volatile. On the contrary secondary storage is non-volatile, has much bigger capacity but is slower (ex. hard disks, solid-state drives).

Each storage system has a device controller and a device driver. The driver provides an uniform interface between the controller and the kernel.

\subsection{Caching}

Caching is a very common technique for speeding up access to commonly used data. Information is temporarily copied from the slower storage to cache and then every time that information is needed the OS will check cache first. Due to the high cost of cache, it is much smaller than other types of storage, therefore cache management must be properly optimized.

\section{Modern system architectures}

Currently most systems are multiprocessor and/or multi-core. In these systems, tasks can be allocated in two ways:
\begin{itemize}
  \item asymmetric processing: each processor/core is assigned a specific task
  \item symmetric processing: each processor/core performs all tasks
\end{itemize}
\image{images/Von Neumann.png}{9cm}{A Von-Neumann architecture}

\subsection{Difference between multiprocessor and multi-core}

Multiprocessor systems have multiple processors with a single CPU and share the same system bus and sometimes the clock. Multi-core systems have a single processor that contains multiple CPUs. Multi-core systems are more widespread because they usually consume less power than multiprocessor systems and because on-chip buses are faster.

\image{images/Multicore.png}{5cm}{A multi-core processor}

\subsection{Clustered systems}

Clustered systems are systems composed of multiple machines that usually share the same storage via a storage-area network (SAN). These systems provide a high-availability service that can survive failures of single machines.
\begin{itemize}
  \item Symmetric clustering: all machines can run tasks and they monitor each
  other. If a machine fails the other can take over.
  \item Asymmetric clustering: each machine is assigned to a specific set of
  tasks. If a machine fails another machine that was turned on and in
  ``hot-standby mode'' takes over.
\end{itemize}

\subsection{Multi-programmed systems}

The OS can run multiple tasks on the same CPU by using a technique called multiprogramming (batch system): the OS organizes jobs so that the CPU has always one ready to execute. When a job has to wait (for example for I/O) the OS switches to another job. This is called job scheduling. Timesharing (multitasking) is an extension of this technique where the OS switches so frequently among different tasks that the user doesn't notice and can interact with all applications at the same time. This is needed for ``window'' based systems, where the user can see multiple things are the same time.

\image{images/Memory layout.png}{4cm}{Memory layout for multiprogrammed systems}

The OS and users share the same hardware, devices and software resources. To protect the system and avoid that different jobs can write in some areas of the memory a privilege system is established. In dual-mode systems jobs can be run in user mode or kernel mode. Some instructions are allowed only for kernel mode systems. For example Intel processors have for modes of operation, where 0 is fully privileged and 3 is fully restricted.

\subsection{Process management}

A process is a program in execution. The \italics{program} is a passive entity, while the \italics{process} is an active entity. The life of the process is generally managed by the operating system. Single-threaded processes have a program counter specifying the location of the next instruction to execute. Instructions are executed sequentially, until the end fo the program is reached. Multi-threaded process has one program counter per thread.

If a system has more cores, each core has its own program counter.

\subsection{Memory management}

To execute a program, the instructions must be in memory. Memory management is
handled by the operating system and has the following goals:
\begin{itemize}
  \item Keeping track of which parts of memory are currently being used and by
  whom
  \item Deciding which processes (or parts thereof) and data to move into and
  out of memory
  \item Allocating and deallocating memory space as needed
\end{itemize}

\subsection{Storage management}

The OS provides a logical view of the storage and abstracts the physical properties in *files*. Files are organized in directories and there usually is an access control system. The OS deals with free-space management, storage allocation and disk scheduling.

The memory is therefore organized in a hierarchy, where each level offers different access speeds. While transferring data from a level to another, the OS must ensure that the data stays consistent. Moreover, multiprocessor environment must provide cache coherency in hardware such that all CPUs have the most recent value in their cache.

\subsection{I/O management}

The OS hides the peculiarities of hardware devices from the user using I/O subsystems. These subsystems are responsible for the device-driver interfaces and memory management of I/O including buffering (storing data temporarily while it is being transferred), caching (storing parts of data in faster storage for performance), spooling (the overlapping of output of one job with input of other jobs).

\subsection{OS protection}

OS must provide mechanisms to defend the system against external attacks. An attack is anything posing a threat to confidentiality, availability or integrity. For example OS distinguish among users, where each has a specific set of privileges. Privilege escalation is an attack where a user can gain privileges of a more privileged user.

\subsection{Computing environments}

There exist many computing environments, such as:
\begin{itemize}
  \item Stand-alone general purpose machines
  \item Network computers (thin clients)
  \item Mobile computers
  \item Real-time embedded systems: operating system that runs processes with
  very important time constraints
  \item Cloud computing
  \item Client-server computing
  \item Peer-to-peer computing
  \item Distributed computing: many systems connected together over a network
  \item Virtualization: guest OS emulates another OS or hardware and runs
  software on it. The program that manages this is called VMM (Virtual machine
  manager).
\end{itemize}
\image{images/Virtualization.png}{10cm}{Virtualization}

\section{Services provided by operating systems}

Operating systems provide the following services:
\begin{itemize}
  \item User interface: can be command-line (CLI), Graphics User Interface
  (GUI), Batch
  \item Program execution - The system must be able to load a program into
  memory and to run that program
  \item I/O operations
  \item File-system manipulation
  \item Communication between processes
  \item Error detection: errors may occur in CPU and memory hardware, in I/O
  devices, in user program
  \item Resource allocation: when multiple users or multiple jobs running
  concurrently, resources must be allocated to each of them
  \item Accounting: to keep track of which users use how much and what kinds
  of computer resources
  \item Protection and security
\end{itemize}
\image{images/Services.png}{12cm}{Services provided by an operating system}

\section{System calls}

System calls are an interface provided by the operating system to interact with it. They are mostly accessed by using a high-level API provided by a language such as C, C++ etc. In this way developers can use a single API that works on all operating systems and leave the actual system call to the underlying library written for that specific platform. The high-level API can also check for errors before calling the system call.

\image{images/C example.png}{6cm}{The \code{printf()} function in C uses the \code{write()} system call to print to the screen}

\subsection{Parameter passing}

A system call usually requires some parameters, for ex. the \code{open_file()} system call needs to know the name of the file. Parameters can be passed using predefined specific registers. Often there are not enough registers for all required parameters, so parameters can be also stored in memory in a table and just the address of the table is put into the register.

\image{images/Parameter passing.png}{10cm}{Parameter passing}

Examples of system calls:
\begin{itemize}
    \item Process management: create, terminate, load, execute, get process attributes, set process attributes, wait time, wait event, signal event, dump memory on error, single step executing for debugging, locks for managing shared data
    \item File management: create, open, delete, read, write
    \item Device management: request device, release device, read, write, get device attributes, set device attributes
    \item Information maintenance: get time or date, set time or date, get system data, set system data
    \item Communications: send/receive messages, open/close connection, gain access to shared memory
    \item Protection: control access to resources, get and set permissions, allow/deny user access
\end{itemize}

\section{OS structure}
OSs may be structured in different ways or may be designed according to different architectures.
\subsection{Simple structure - MS-DOS}
MS-DOS has a very simple structure: a shell starts a program and when the process ends the shell is rebooted into a new program. There is at most one process running.
\subsection{Monolithic kernel - UNIX}
Originally UNIX had a monolithic structure. The kernel provided a large number of functions, such as the file system, CPU scheduling, memory management. The advantages of using a monolithic kernel are that it is fast and energy-efficient, but it is not modular and even small changes require refactoring of the code and recompilation of the whole OS.
\subsection{Layered approach}
The operating system is divided into multiple layers, where each layer is built on top of the lower layers (similar to ISO/ISO reference model and TCP/IP stack). This allows for more modularity and a change of one layer doesn't always imply a recompilation of the whole operating system. An example of a possible layered structure is the following: hardware -> drivers -> file system -> error detection and protection -> user programs.
\subsection{Microkernel}
The microkernel approach moves processes as much as possible outside the kernel into the user space. Communication between modules is achieved using message passing. The advantages of this approach are full modularity and extendability, security (a malicious process can't damage others) and reliability (less code is running in kernel mode). Message passing introduces additional overhead, thus has a negative performance impact.
\image{Images/Microkernel structure.png}{10cm}{Microkernel structure}