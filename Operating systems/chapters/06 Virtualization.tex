\chapter{Virtualization}
The idea of virtualization is to abstract the access to the hardware, so that multiple operating systems can run at the same time. The software in charge of that is the virtual machine manager (VMM) or hypervisor. The OS that interfaces directly to the hardware is called host, while the ones running through a VMM are called guests. There are different types of VMMs:
\begin{itemize}
    \item Type 0 VMM: the VMM is implemented directly in hardware
    \item Type 1 VMM: the VMM is an OS-like program which allows execution of virtual machines and it runs at boot
    \item Type 2 VMM: the VMM is run as a normal user process on a real operating system. This solution is the one that gives the less freedom, since the host OS regulates the access to resources, the processor used, the amount of CPU time etc.
\end{itemize}

There are other variations of VMMs, namely:
\begin{itemize}
    \item Paravirtualization: the guest operating system is optimized to work well with the VMM
    \item Emulators: allow applications written for one hardware
    environment to run on a very different hardware environment
    \item Application containment: not actual virtualization, provides virtualization-like features by segregating applications from the operating system, making them more secure,
    manageable
\end{itemize}

\subsubsection{Running Mode}
The VMM is run in user mode on the host OS, but the guest OS need s to run some software in kernel mode. This means that the VMM needs to virtualize the kernel mode, which is known as trap-and-emulate.

\section{Containers}
Containers are a form of operating system virtualization. Inside a container are all the necessary executables, binary code, libraries, and configuration files to execute a software. The main difference with virtual machines is that they do not contain an OS. They still require a translation layer to interface with the host operating system, such as the Docker engine.

\section{Other applications of virtualization}
Another application of the virtualization idea is the programming environment virtualization. In this case the programming language is designed to run within a custom-built virtualized environment. For example Java has many features that depend on running in Java Virtual Machine (JVM).